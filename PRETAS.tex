\begin{itemize}
\item \textbf{Filosofia black bloc} volta o olhar para quando, em junho de
2013, em meio à maior erupção social das últimas décadas, o black
bloc ganhou os holofotes como nova prática de luta e manifestação. Analistas
à direita e à esquerda foram impelidos a abordar o movimento,
quase sempre municiando um repertório conceitual incompatível com os
significados do black bloc, sem compreendê-lo em seus próprios termos.
Murilo Duarte Costa Corrêa, com vastas referências sobretudo no pós"-estruturalismo
francês, procura suprir essa carência e conceber um arcabouço
teórico que permita abordar o black bloc como fenômeno. Produzir, no
pensamento, uma filosofia black bloc.


\item \textbf{Murilo Duarte Costa Corrêa} é Professor Adjunto de Teoria
Política na Faculdade de Direito da Universidade Estadual de Ponta Grossa
(\versal{DDE/UEPG}), além de Affiliated Researcher na Faculty of Law and Criminology
da Vrije Universiteit Brussel, onde realizou estágio de pós"-doutorado com
pesquisa sobre a filosofia do campo social de Gilles Deleuze. Assina como autor
\emph{Direito e Ruptura: ensaios para uma filosofia do direito na
imanência}, e é também organizador de \emph{O estado de exceção e as formas jurídicas}
e \emph{Pensar a Netflix: séries de pop filosofia e política}, se destacando
em pesquisas interdisciplinares nas áreas do Direito, da Filosofia, da Teoria
Social e da Teoria Política.
\end{itemize}

