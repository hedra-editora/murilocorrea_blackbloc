\begin{itemize}
\item \textbf{2013: memórias e resistências} é fruto dos levantes populares que tomaram as ruas em 2013. Aqui estão compilados memórias, relatos, desabafos, entrevistas, análises e comunicações públicas da professora Camila Jourdan, uma das 23 pessoas processadas quando dos protestos contra a Copa do Mundo. De temática anarquista e insurgente, seus textos se pretendem um registro da história recente do país e um contradiscurso na disputa do que significou e legou 2013.
  
\item \textbf{Camila Jourdan} é professora do Departamento de Filosofia da \versal{UERJ},
atuando sobretudo nas áreas de Filosofia da Linguagem, Teoria do Conhecimento, Lógica
e Filosofia Contemporânea. Em 2013, participou ativamente dos levantes populares que
tomaram o Rio de Janeiro e o Brasil, e, tendo sido eleita pela mídia e pelo Estado como
uma das ``organizadoras'' dos protestos, sofreu perseguições legais incessantes que culminaram
na sua condenação à prisão, em julho de 2018.
\end{itemize}

