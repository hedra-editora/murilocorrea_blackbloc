\chapter*{}

\begin{flushright}
\emph{Acácio Augusto \& Renato Rezende}
\end{flushright}

\medskip

\noindent{}A coleção \emph{Ataque} irrompe sob efeito de junho de 2013.
Esse acontecimento recente da história das lutas sociais no Brasil, a um só
tempo, ecoa combates passados e lança novas dimensões para os
enfrentamentos presentes. O critério zero da coleção é o choque com os
poderes ocorrido durante as \emph{jornadas de
junho}, mas não só. Busca"-se captar ao menos uma pequena parte do fluxo de
radicalidade (anti)política que escorre pelo planeta a despeito da
tristeza cívica ordenada no discurso da esquerda institucionalizada. Um
contrafluxo ao que se convencionou chamar de onda conservadora. Os
textos reunidos são, nesse sentido,
anárquicos, mas não apenas de autores e temas ligados aos
anarquismos. Versam sobre batalhas de
rua, grupos de enfrentamento das forças policiais, demolição da forma"-prisão que
ultrapassa os limites da prisão"-prédio. Trazem também análises sobre os
modos de controle social e sobre o terror do racismo de Estado. Enfim, temas de enfrentamento com
escritas que possuem um alvo. 
O nome da coleção foi tomado de um antigo
selo punk de São Paulo que, em 1985, lançou a coletânea \emph{Ataque
Sonoro}. Na capa do disco dois mísseis, um soviético e outro
estadunidense, apontam para a cidade de São Paulo, uma metrópole do que
ainda se chamava de terceiro mundo. Um anúncio, feito ao estilo audaz
dos punks, do que estava em jogo: as forças rivais atuam juntas contra o
que não é governado por uma delas. Se a configuração mudou de lá para
cá, a lógica e os alvos seguem os mesmos. Diante das mediações e
identidades políticas, os textos desta coleção optam pela tática do
ataque frontal, conjurando as falsas dicotomias que organizam a
estratégia da ordem. Livros curtos para serem levados no bolso, na
mochila ou na bolsa, como pedras ou coquetéis molotov.
Pensamento"-tática que anima o enfrentamento colado à urgência do
presente. Ao serem lançados, não se espera desses livros mais do que
efeitos de antipoder, como a beleza de exibições pirotécnicas. Não há
ordem, programa, receita ou estratégia a serem seguidos. Ao atacar
radicalmente a única esperança possível é que se perca o controle e,
como isso, dançar com o caos dentro de si. Que as leituras produzam
efeitos no seu corpo.

%\section*{Títulos:}%

%\begin{Parskip}
%Camila Jourdan. \emph{Chocolate e gás lacrimogênio: luta, prisão e
%  resistência em 2013}.%

%Acácio Augusto. \emph{Desde junho: anarquia, antipolítica e terror de
%  Estado no Brasil}.%

%Murilo Duarte Costa Correa. \emph{Filosofia black bloc.}%

%Carlos Taibo. \emph{Repensar a anarquia. Ação direta, autogestão e
%  autonomia.} %(Traduzir do espanhol)%

%Gustavo Simões. \emph{O desconcerto anarquista de Jonh Cage.}%

%José Oiticica. \emph{Ação direta. Meio século de pregação libertária}.%

%Mark Bray. \emph{Antifa. Um manual antifascista}. %(Traduzir do inglês)%

%Hakim Bey. \emph{Caos. Terrorismo poético e outros crimes exemplares.}%

%Vários Autores. \emph{Anarquistas e as prisões}. %(Coletânea a ser preparada)%

%Richard Day. \emph{Gramsci is dead. Anarquia e movimentos pós"-Seattle
%  1999.}%

%Voltarine De Claire. \emph{Ação direta}. %(tradução do inglês, segue em espanhol)
%\end{Parskip}
